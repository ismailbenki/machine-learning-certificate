\documentclass[12pt,a4paper]{article}

\usepackage[T1]{fontenc} 
\usepackage[utf8]{inputenc}

\usepackage[a4paper]{geometry}
\geometry{hscale=0.85,vscale=0.85,centering}
\usepackage{lmodern}
\usepackage{framed}
\usepackage[framed]{ntheorem}
\usepackage{xcolor}
\usepackage{graphicx}
\graphicspath{ {./img/} }
\usepackage{amssymb}
\usepackage{amsmath}
\usepackage{amsfonts}
\usepackage{dsfont}
\usepackage{young}
\usepackage[document]{ragged2e}
\usepackage[vcentermath]{youngtab}
\usepackage{listings}
\usepackage{titlesec}
\usepackage{pgfplots}
\pgfplotsset{compat=1.12}
\usepackage{url}
\usepackage{hyperref}
\hypersetup{					% setup the hyperref-package options
	breaklinks=true,			% 	- allow line break inside links		%
	colorlinks,
	citecolor=black,
	filecolor=black,
	linkcolor=black,
	urlcolor=black
}
\usepackage{tikz}
\usepackage{subfigure}
\usepackage[francais]{babel}



\titleformat{\section}
{\centering \Large \normalfont \scshape}{\thesection}{1em}{}
\titleformat{\subsection}
{\centering \large \scshape}{\thesubsection}{1em}{}
\titleformat{\subsubsection}
{ \normalsize \scshape}{\thesubsubsection}{1em}{}
\numberwithin{equation}{section}

\setcounter{tocdepth}{2}

\newtheorem{theorem}{Théorème}[section]
\newtheorem{coro}{Corollaire}[section]
\newtheorem{lem}{Lemme}[section]

\newcommand{\T}{\mathfrak{T}}
\renewcommand{\Pr}{\mathbb{P}}
\renewcommand{\epsilon}{\varepsilon}
\newcommand{\E}{\mathbb{E}}
\newcommand{\bP}{\mathbf{P}}
\newcommand{\bE}{\mathbf{E}}
\newcommand{\gSym}{\mathfrak{S}_N}
\newcommand{\lN}{\ell_N}
\newcommand{\C}{\mathbb{C}}
\newcommand{\Tr}{\mathrm{Tr}}
\newcommand{\cqfd}{\begin{flushright} $\Box$ \end{flushright}}
\newcommand{\preuve}{$Preuve.$ }


\title{\scshape \huge Forest Cover Type Prediction}
\author{\textbf{Kevin Zagalo} \\  \url{kevin.zagalo@etu.upmc.fr}  \and \textbf{Ismail Benkirane} \\ \url{ismail.benkirane@etu.upmc.fr}}
\date{}

\begin{document}

	\maketitle
	
{\small Projet pour le cours \textit{Apprentissage Statistique} du LIP6, Sorbonne Université} \hfill Janvier 2019
	
	\hrulefill

	{\small \justify Ce projet a pour but de proposer et tester des modèles pour l'étude de la base de données \textit{Covertype} \footnote{\url{https://archive.ics.uci.edu/ml/datasets/Covertype}}, de 581\,012 instances, avec 54 attributs et 7 classes à prédire, sans données manquantes.\\
	Les attributs sont les suivants : \\
	
		\begin{tabular}{l l p{5.2cm}}
			Nom & Unité & Description \\
			\hline
			\verb!Elevation! & mètres & Altitude \\ 
			\verb!Aspect! & degrés & Orientation \\ 
			\verb!Slope! & degrés & Pente \\
			\verb!Horizontal_Distance_To_Hydrology! & mètres & Distance horizontale au point d’eau le plus proche\\
			\verb!Vertical_Distance_To_Hydrology! & mètres & Distance verticale au point d’eau le plus proche \\
			\verb!Horizontal_Distance_To_Roadways! & mètres & Distance horizontale à la route la plus proche \\
			\verb!Hillshade_9a!m & entier entre 0 et 255 & Ombrage à 9h au solstice d’été \\
			\verb!Hillshade_Noon! & entier entre 0 et 255 & Ombrage à 12h au solstice d'été\\
			\verb!Hillshade_3pm! & entier entre 0 et 255 &  Ombrage à 15h au solstice d'été \\
			\verb!Horizontal_Distance_To_Fire_Points! & mètres & Distance horizontale au départ de feu le plus proche\\
			\verb!Wilderness_Area! & 4 colonnes binaires & Wilderness area designation \\
			\verb!Soil_Type! & 40 colonnes binaires & Type de sol\\
			\verb!Cover_Type! & entier entre 1 et 7 & Classe\\
		\end{tabular}\\
	
	\medskip
	Il s'agit donc d'un problème de classification multi-classe avec 7 classes.
	}
	\medskip
	
	\hrulefill

	\tableofcontents
	
	\newpage
	
	\section{Chargement des données}
	
	\section{Analyse préliminaire et pré-traitement des données}
	
	\section{Test de différents modèles}
	
	\end{document}